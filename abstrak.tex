\chapter{ABSTRAK}
\label{chp:abstrak}
\vspace{\baselineskip}
\begin{center}
  \bfseries\large
  \MakeUppercase{\titlename} \\
  \vspace{1\baselineskip}
  \normalsize\normalfont Oleh \\
  \bfseries\large \authorname \\
  NIM: \authorid \\
  (\department) \\
\end{center}

\vspace{2\baselineskip} \noindent
Abstrak merupakan penjelasan singkat dan padat tentang pekerjaan dan hasil penelitian TA, yang dituliskan secara teknis. Abstrak memiliki karakter tegas dan komprehensif, dan hanya dapat dituliskan setelah pekerjaan penelitian telah mencapai tahap tertentu, dan karenanya ada hasil penelitian yang dapat dilaporkan. Abstrak ditulis menjelang akhir penyelesaian penulisan buku TA.
\vspace{\baselineskip}

\noindent
Secara umum, abstrak memuat beberapa komponen penting, yaitu: konteks atau cakupan pekerjaan penelitian, tujuan penelitian, metodologi yang digunakan selama penelitian, hasil-hasil penting yang dapat ditambahkan dengan implikasinya, dan simpulan dari penelitian. Dengan demikian, suatu abstrak tidak dapat dituliskan apabila penelitian belum mencapai hasil tertentu, apalagi kalau penelitiannya pun belum dilakukan.
\vspace{\baselineskip}

\noindent
Panjang abstrak sebaiknya dicukupkan dalam satu halaman, termasuk kata kunci. Tiga kata kunci dipandang cukup, yang masing-masingnya memuat paduan kata utama, yang dapat merepresentasikan isi Abstrak. Halaman Abstrak tidak memuat informasi judul dan penulis, sehingga tidak secara langsung dapat digunakan sebagai lembaran Abstrak Sidang TA yang disediakan untuk hadirin, yang memerlukan tambahan (sekurangnya) dua informasi tersebut.

\begin{flushleft}
  Kata kunci: Konsep Abstrak, Komponen Abstrak, Kata Kunci.
\end{flushleft}