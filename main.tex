\documentclass[fleqn,a4paper,12pt,oneside,onecolumn]{book}

% layout margin
\usepackage[inner=4cm, outer=3cm, top=3cm, bottom=3cm]{geometry}
% font family 'times new roman'-like, PLEASE REMOVE THESE MADNESS!
\usepackage{newtxtext,newtxmath}
% Package terkait AMS dan matematika
\usepackage{amsmath}
% load image package
\usepackage{graphicx}
% hanging indent
\usepackage{hanging}
% chapter style centering
\usepackage[explicit]{titlesec}
\titleformat{\chapter}[hang]
    {\centering\large\bfseries}
    {\chaptertitlename\ \thechapter\;\;\;} 
    {0pt}
    {\centering\MakeUppercase{#1}}
\titlespacing{\chapter}{0pt}{-24pt}{*0}
% section size
\titleformat{\section}
    {\normalsize\bfseries}
    {\thesection\;\;\;} 
    {0pt}
    {#1}

% TODO: page numbering position center bottom just about 1.5cm
% page style
\pagestyle{plain}
% roman chapter numbering
\renewcommand{\thechapter}{\Roman{chapter}} 
% package bahasa
\usepackage[english]{babel} % TODO: nanti ganti ke indo...
% \usepackage[indonesian]{babel}
% fancy header, untuk mengganti tampilan header
\usepackage{fancyhdr}
% Package untuk menampilkan daftar pustaka
% TODO: add indonesian format for biblatex, see https://tex.stackexchange.com/questions/200932/what-is-the-most-appropriate-way-to-configure-biblatex-for-use-with-an-unsupport
\usepackage[
backend=biber,
style=numeric,
sorting=nyt
]{biblatex}
\usepackage{csquotes}
\addbibresource{sample.bib}
% Package hyperlinks dokumen
\usepackage{hyperref}
% indent first paragraph
\usepackage{indentfirst}
%----------------------------------------
% JUDUL DALAM BAHASA INDONESIA
\AtBeginDocument{\renewcommand{\contentsname}{DAFTAR ISI}}
\AtBeginDocument{\renewcommand{\listfigurename}{DAFTAR GAMBAR}}
\AtBeginDocument{\renewcommand{\listtablename}{DAFTAR TABEL}}
\AtBeginDocument{\renewcommand{\chaptername}{BAB}}
\AtBeginDocument{\renewcommand{\bibname}{DAFTAR PUSTAKA}}
% \AtBeginDocument{\renewcommand{\appendixpagename}{LAMPIRAN}}
% \AtBeginDocument{\renewcommand{\appendixname}{LAMPIRAN}}
% \AtBeginDocument{\renewcommand{\appendixtocname}{LAMPIRAN}}
\AtBeginBibliography{\vspace*{\baselineskip}}
%
\renewcommand{\thefootnote}{\arabic{footnote}}
\renewcommand{\index}{\arabic{indeks}}
%----------------------------------------
%-----------------------------------------
% month abbreviation; Please remove this abomination if possible 😡
\usepackage{xifthen}
\newcommand{\ifequals}[3]{\ifthenelse{\equal{#1}{#2}}{#3}{}}
\newcommand{\case}[2]{#1 #2} % Dummy, so \renewcommand has something to overwrite...
\newenvironment{switch}[1]{\renewcommand{\case}{\ifequals{#1}}}{}
\newcommand{\monthabv}[2]{
    \begin{switch}{#1}
        \case{Januari}{Jan}
        \case{Februari}{Feb}
        \case{Maret}{Mar}
        \case{April}{Apr}
        \case{Mei}{Mei}
        \case{Juni}{Jun}
        \case{Juli}{Jul}
        \case{Agustus}{Ags}
        \case{September}{Sep}
        \case{Oktober}{Okt}
        \case{November}{Nov}
        \case{Desember}{Des}
        #2
    \end{switch}
}
%-----------------------------------------
% custom command
\newcommand{\titlename}{Menggunakan GAIA DR3 Untuk Mengkonstrain Kerapatan Materi Gelap di Dekat Matahari dalam Kerangka MOND}
\newcommand{\authorname}{Muhammad Ali Syaifudin}
\newcommand{\authorid}{20321005}
\newcommand{\department}{Program Studi Magister Astronomi}
\newcommand{\completionyear}{2023}
\newcommand{\completionmonth}{Januari}
\newcommand{\completiondate}{\monthabv{\completionmonth}{\completionyear}}
\newcommand{\fullcompletiondate}{5 \completionmonth\ \completionyear}
%-----------------------------------------
\begin{document}
% Bagian Awal Buku
\frontmatter
\clearpage
\begin{titlepage}
  \label{chp:judul}
  \centering\bfseries\large
  \MakeUppercase{\titlename} \\
  \vspace{\stretch{2}}
  TESIS \\
  \vspace{\baselineskip}
  \normalsize Karya tulis sebagai salah satu syarat \\
  untuk memperoleh gelar Magister dari \\
  Institut Teknologi Bandung\\
  \vspace{\stretch{1}}
  Oleh \\ \large
  \MakeUppercase{\authorname} \\
  NIM: \authorid \\
  (\department) \\
  \vspace{\stretch{1}}
  \begin{figure}[!h]
    \centering
    \includegraphics[width=2.35cm,height=3.5cm]{gbr/gajah.jpg}
  \end{figure}
  \vspace{\stretch{1.5}}
  INSTITUT TEKNOLOGI BANDUNG \\
  \completiondate
\end{titlepage}
\pagenumbering{roman}
\chapter{ABSTRAK}
\label{chp:abstrak}
\vspace{\baselineskip}
\begin{center}
  \bfseries\large
  \MakeUppercase{\titlename} \\
  \vspace{1\baselineskip}
  \normalsize\normalfont Oleh \\
  \bfseries\large \authorname \\
  NIM: \authorid \\
  (\department) \\
\end{center}

\vspace{2\baselineskip}
Abstrak merupakan penjelasan singkat dan padat tentang pekerjaan dan hasil penelitian TA, yang dituliskan secara teknis. Abstrak memiliki karakter tegas dan komprehensif, dan hanya dapat dituliskan setelah pekerjaan penelitian telah mencapai tahap tertentu, dan karenanya ada hasil penelitian yang dapat dilaporkan. Abstrak ditulis menjelang akhir penyelesaian penulisan buku TA.
\vspace{\baselineskip}

Secara umum, abstrak memuat beberapa komponen penting, yaitu: konteks atau cakupan pekerjaan penelitian, tujuan penelitian, metodologi yang digunakan selama penelitian, hasil-hasil penting yang dapat ditambahkan dengan implikasinya, dan simpulan dari penelitian. Dengan demikian, suatu abstrak tidak dapat dituliskan apabila penelitian belum mencapai hasil tertentu, apalagi kalau penelitiannya pun belum dilakukan.
\vspace{\baselineskip}

Panjang abstrak sebaiknya dicukupkan dalam satu halaman, termasuk kata kunci. Tiga kata kunci dipandang cukup, yang masing-masingnya memuat paduan kata utama, yang dapat merepresentasikan isi Abstrak. Halaman Abstrak tidak memuat informasi judul dan penulis, sehingga tidak secara langsung dapat digunakan sebagai lembaran Abstrak Sidang TA yang disediakan untuk hadirin, yang memerlukan tambahan (sekurangnya) dua informasi tersebut.

\begin{flushleft}
  Kata kunci: Konsep Abstrak, Komponen Abstrak, Kata Kunci.
\end{flushleft}
\chapter{\itshape ABSTRACT}
\label{chp:abstract}
\vspace{\baselineskip}
\begin{center}
  \bfseries\large\itshape
  \MakeUppercase{\titlename} \\
  \vspace{1\baselineskip}
  \normalsize\normalfont\itshape By \\
  \bfseries\large\normalfont \authorname \\
  NIM: \authorid \\
  (\itshape\department) \\
\end{center}
\itshape
\vspace{2\baselineskip} \noindent
Abstrak merupakan penjelasan singkat dan padat tentang pekerjaan dan hasil penelitian TA, yang dituliskan secara teknis. Abstrak memiliki karakter tegas dan komprehensif, dan hanya dapat dituliskan setelah pekerjaan penelitian telah mencapai tahap tertentu, dan karenanya ada hasil penelitian yang dapat dilaporkan. Abstrak ditulis menjelang akhir penyelesaian penulisan buku TA.
\vspace{\baselineskip}

\noindent
Secara umum, abstrak memuat beberapa komponen penting, yaitu: konteks atau cakupan pekerjaan penelitian, tujuan penelitian, metodologi yang digunakan selama penelitian, hasil-hasil penting yang dapat ditambahkan dengan implikasinya, dan simpulan dari penelitian. Dengan demikian, suatu abstrak tidak dapat dituliskan apabila penelitian belum mencapai hasil tertentu, apalagi kalau penelitiannya pun belum dilakukan.
\vspace{\baselineskip}

\noindent
Panjang abstrak sebaiknya dicukupkan dalam satu halaman, termasuk kata kunci. Tiga kata kunci dipandang cukup, yang masing-masingnya memuat paduan kata utama, yang dapat merepresentasikan isi Abstrak. Halaman Abstrak tidak memuat informasi judul dan penulis, sehingga tidak secara langsung dapat digunakan sebagai lembaran Abstrak Sidang TA yang disediakan untuk hadirin, yang memerlukan tambahan (sekurangnya) dua informasi tersebut.

\begin{flushleft}
  Keywords: Konsep Abstrak, Komponen Abstrak, Kata Kunci.
\end{flushleft}\normalfont
\chapter{\phantom{PENGESAHAN}}
\label{chp:pengesahan}
\begin{center}
  \bfseries\large
  \MakeUppercase{\titlename} \\
  \vspace{2\baselineskip}
  \normalfont Oleh \\
  \textbf{\authorname \\
    NIM: \authorid \\
    (\department)} \\
  \vspace{\baselineskip}
  Institut Teknologi Bandung \\
  \vspace{3\baselineskip}
  Menyetujui \\
  Dosen Pembimbing \\
  \vspace{\baselineskip}
  Tanggal 5 Agustus 2020 \\
  \vspace{5\baselineskip}
  \rule{5cm}{1pt} \\
  Pak Ikbal
\end{center}

\vspace{0.5cm} % Angka ini tidak diubah

\begin{flushleft}
  \textbf{Tim Penguji:\\
    1. Nama Penguji 1\\ % Nama dan gelar
    2. Nama Penguji 2\\ % Nama dan gelar
    3. Nama Penguji 3}\\ % Nama dan gelar
\end{flushleft}
\chapter{PEDOMAN PENGGUNAAN TESIS}
\label{chp:pedoman}
\vspace{\baselineskip}
Tesis Magister yang tidak dipublikasikan terdaftar dan tersedia di Perpustakaan Institut Teknologi Bandung, dan terbukan untuk umum dengan ketentuan bahwa hak cipta ada pada penulis dengan mengikuti aturan HaKI yang berlaku di Institut Teknologi Bandung. Referensi kepustakaan diperkenankan dicatat, tetapi pengutipan atau peringkasan hanya dapat dilakukan seizin penulis dan harus disertai dengan kaidah ilmiah untuk menyebutkan sumbernya.
\vspace{\baselineskip}

Sitasi hasil penelitian Tesis ini dapat ditulis dalam bahasa Indonesia sebagai berikut: \\

\begin{hangparas}{.25in}{1}
  Nama Belakang, Inisial Nama Depan. (Tahun): \emph{Judul tesis}, Tesis Program Magister, Institut Teknologi Bandung.
\end{hangparas}

\vspace{\baselineskip}

\noindent
dan dalam bahasa Ingris sebagai berikut:
\vspace{\baselineskip}

\begin{hangparas}{.25in}{1}
  Nama Belakang, Inisial Nama Depan. (Tahun): \emph{Judul tesis yang telah diterjemahkan dalam bahasa Inggris}, Master's Thesis, Institut Teknologi Bandung.
\end{hangparas}
\vspace{\baselineskip}

Memperbanyak atau menerbitkan sebagian atau seluruh tesis haruslah seizin Dekan Sekolah Pascasarjana, Institut Teknologi Bandung.
\chapter*{}
\addcontentsline{toc}{chapter}{HALAMAN PERUNTUKAN}
\label{chp:dedikasi}
\begin{center}

  \vspace{3cm} % Angka ini dapat diubah sesuai keperluan

  \begin{quote} % Selalu gunakan format quote
    \textit{\large{Untuk keluarga besar Ganesha Badak Singa}}\\
    \vspace{0.2cm} % Angka ini dapat diubah sesuai keperluan
    \textit{Untuk Tuhan, Bangsa, dan Almamater}\\
    \texttt{In Harmonia Progressio}\\
    \vspace{0.5cm} % Angka ini dapat diubah sesuai keperluan
    \dots\\
    \textit{I'm a shooting star leaping through the skies\\
      Like a tiger defying the laws of gravity\\
      I'm a racing car passing by like Lady Godiva\\
      I'm gonna go go go\\
      There's no stopping me}\\
    \vspace{0.5cm}
    \textit{I'm burning through the skies Yeah!\\
      Two hundred degrees\\
      That's why they call me Mister Fahrenheit\\
      I'm trav'ling at the speed of light\\
      I wanna make a supersonic man of you\\
      \dots}\\
    \vspace{0.5cm}
    \footnotesize{\#Don't Stop Me Know --Queen}
  \end{quote}

\end{center}


\chapter{KATA PENGANTAR}
\label{chp:pengantar}
\vspace{\baselineskip}

Halaman kata pengantar dicetak pada halaman baru. Pada halaman ini mahasiswa S2 berkesempatan untuk menytakan terima kasih secara tertulis kepada pembimbing dan perorangan lainnya yang telah memberi bimbingan, nasihat, saran, dan kritik, serta kepada mereka yang telah membantu melakukan penelitian, kepada perorangan atau badan yang telah memberi bantuan pembiayaan, dan sebagainya.
\vspace{\baselineskip}

Cara menulis kata pengantar beraneka ragam, tetapi semuanya hendaknya menggunakan kalimat yang baku. Ucapan terima kasih agar dibuat tidak berlebihan dan dibatasi hanya yang ``\emph{scientifically related}''.
% Daftar-daftar
\tableofcontents
\listoffigures
\listoftables
\chapter{DAFTAR SINGKATAN}
\label{chp:singkatan}
\vspace{1.0cm}
% Entri pada Daftar Singkatan ini hanya untuk contoh
\begin{table}[h!]
  \small
  %\caption[Singkatan]{Daftar singkatan yang digunakan.}
  \label{tab:singkatan}
  \begin{center}
    \begin{tabular}{ll}
      \hline
      \textbf{Singkatan} & \textbf{Arti}                        \\
      \hline
      2MASS              & $\textit{Two-micron All Sky Survey}$ \\
      CCD                & $\textit{Charge-Coupled Device}$     \\
      HMXB               & $\textit{High Mass X-Ray Binary}$    \\
      LMC                & $\textit{Large Magellanic Cloud}$    \\
      SA                 & Satuan Astronomi                     \\
      \hline
    \end{tabular}
  \end{center}
\end{table}


% Bagian Utama Buku
\mainmatter
\input{bab1.tex}
\chapter{Dasar Teori}
\label{chp:2}
\vspace{\baselineskip}
%-----------------------------------------------------------------------------%	 
\section{Contoh Format Penulisan Judul Anak Bab dengan Panjang Lebih dari Satu Baris} \label{latar-belakang2}
Bagian ini mendeskripsikan gambaran umum, konteks, dan posisi penelitian TA dalam konstelasi perkembangan pengetahuan yang telah dicapai. Penjelasan yang dituliskan menjadi penting karena dengan landasan yang kuat, maka pekerjaan penelitian dapat terarah dilakukan. Hal ini lebih spesifik dan tegas disampaikan pada sub-sub bab berikutnya.

Beberapa pustaka utama yang berperan dominan dapat disampaikan di sini untuk memberi gambaran tentang letak penelitian TA dalam konstelasi keilmuan yang dicapai. Hasil-hasil dari pustaka terbaru dapat menopang Latar Belakang ini menjadi lebih kuat.

Sangat wajar apabila isi sub bab setelah Latar Belakang ini mengalami penyesuaian saat sejumlah hasil penelitian sudah diperoleh dan dianalisis. Pada dasarnya, hal ini dimungkinkan apabila ada penyesuaian kecil, karena fokus penelitian sejatinya sudah jelas sedari awal, namun hasil-hasil yang diperoleh dapat memperbaharui beberapa butir isi sub bab. Oleh karena itu, finalisasi isi Pendahuluan ini biasanya dilakukan menjelang akhir pembuatan laporan penelitian yang dituangkan dalam buku TA.

%-----------------------------------------------------------------------------%	 
\section{Rumusan dan Batasan Masalah} \label{rumusan-batasan-masalah2}
Bagian ini menjadi salah satu bagian penting dalam Pendahuluan. Setelah paparan Latar Belakang, maka masalah yang diangkat pada pekerjaan penelitian perlu dirumuskan dengan baik. Perumusan ini sebaiknya dibahasakan tidak dalam bentuk kalimat pertanyaan, melainkan kalimat aktif, dan dapat memuat lebih dari satu rumusan.

Sejalan dengan ini, setiap masalah yang diangkat selalu memiliki batas. Ada batasan, asumsi, atau kriteria yang menjadi pembatas atas masalah yang diangkat dalam penelitian TA, sehingga arah penelitian dapat fokus. Batasan ini perlu dituliskan secara tegas, dan dapat saja memuat lebih dari satu.

%-----------------------------------------------------------------------------%
\section{Tujuan} \label{tujuan2}
Bagian ini secara tegas menuliskan tujuan pekerjaan penelitian TA, yang dapat memuat lebih dari satu. Pemilihan kata kerja pada Tujuan ini sangat penting karena menggambarkan arah fokus dari jalinan upaya yang dilakukan.

%-----------------------------------------------------------------------------%
\section{Metodologi} \label{metodologi2}
Di sini disampaikan metodologi yang diterapkan pada pekerjaan penelitian TA. Beberpa di antaranya adalah pengamatan dan akuisisi data, eksperimen numerik, studi pustaka, teoretik atau analitik, dan semi analitik dengan komplemen numerik.

%-----------------------------------------------------------------------------%
\section{Sistematika Penulisan} \label{sistematika-penulisan2}
Bagian ini adalah penutup Bab I yang menyampaikan secara ringkas isi setiap bab. Karena pembaca sudah sampai akhir Bab I, yang berarti sudah mengetahui isinya, maka tidak perlu ditulis lagi rincian Bab I. Sebaiknya langsung dituliskan secara ringkas isi rincian bab-bab selanjutnya, misalnya, \textit{Setelah Pendahuluan pada Bab I ini, Bab II akan mengulas tentang \dots}.

Apabila diperlukan, dapat dituliskan konvensi khusus yang digunakan pada penulisan naskah buku TA ini, misalnya tanda titik menggantikan tanda desimal karena alasan kemudahan dan kejelasan dalam formulasi matematika.
% Bagian Akhir Buku
\backmatter
\printbibliography
\end{document}